\documentclass{article}
\usepackage[utf8]{inputenc}
\usepackage[brazilian]{babel}
\usepackage{minted}
\usepackage{natbib}
\usepackage{graphicx}

\title{Agente inteligente para o jogo Gomoku}
\author{Lucas Pedro Bordignon \and Ranieri Schroeder Althoff}
\date{\today}


\begin{document}

\maketitle

\section{Gomoku}
Gomoku, também conhecido como jogo ligue cinco, consiste em, um tabuleiro 15x15, montar uma sequência de 5 peças na
vertical, horizontal ou diagonal, sendo que jogador executa um
movimento por rodada. A Figura \ref{fig:gomoku} mostra um exemplo do tabuleiro do Gomoku.

\begin{figure}[h!]
  \centering
  \includegraphics[scale=1.5]{gomoku}
  \caption{Tabuleiro do jogo}
  \label{fig:gomoku}
\end{figure}

\section{Abordagem}
Dado que o jogo depende de dois agentes para progredir (no nosso caso, um humano, e uma máquina), podemos modelar o
grafo que representa o jogo da seguinte forma:

\begin{itemize}
  \item Cada nodo do grafo representa a situação atual do tabuleiro, em determinado momento do jogo
  \item Cada nodo indica quem é o próximo a jogar
  \item Cada aresta representa uma jogada, ou seja, uma mudança no estado do tabuleiro
\end{itemize}

\section{Algoritmo MiniMax}
O algoritmo a ser utilizado para o agente, utilizando o conceito de podas alpha-beta, em linhas gerais, é o seguinte:

\begin{minted}[escapeinside=||,mathescape=true]{python}
  def min_max_with_pruning(
                          node,
                          beta = +|$\infty$|,
                          alpha = -|$\infty$|):
      # Realiza a poda
      if(beta <= alpha):
          return

      # Nodos folha
      if(node.children.size == 0):
        node.weight = utility_function(node)
        return

      # Segue a busca nos nodos filhos
      for children in node:
          min_max_with_pruning(children, beta, alpha)
          if(node.turn == 'Human'):
              beta = node.weight if(node.weight < beta)
          elif(node.turn == 'Machine'):
              alpha = node.weight if(node.weight > alpha)
\end{minted}


\section{Heurística e Função de utilidade}
  \subsection {Funcão de utilidade}
    A função de utilidade deve levar em consideração diversas combinações de jogadas e disposições das peças
    no tabuleiro do jogo, bem como o resultado de uma sequência de jogadas. Em nosso agente, para situações de
    vitória, empate e derrota, o resultado da função recebe um acréscimo ou decréscimo, baseado no resultado do jogo.

    Consequentemente, quando tivermos uma vitória, somamos a quantidade de espaços vazios deixados no tabuleiro ao
    resultado, dado que quanto menor a quantidade de movimentos necessários para uma vitória, melhor é esse conjunto
    de jogadas. No caso de uma derrota, descontamos a quantidade de espaços vazios, pois uma derrota forçando o
    adversário a melhorar e tomar a melhor decisão em cada rodada é mais vantajosa do que uma simples derrota
    rápida, com o tabuleiro quase vazio.

    Além disso, a função modela situações que favorecem mais a máquina (agente) em uma menor quantidade de jogadas.
    Por exemplo, é mais vantajoso para o mesmo possuir uma sequência de três peças no tabuleiro(tripla) do que $n$
    sequências de duas peças(dupla), dado que uma tripla está mais próximo da vitória, independente do valor de $n$.

    Realizando algunas cálculos, descobrimos que o máximo número de duplas que um jogador consegue realizar, sem
    montar uma tripla, gira em torno de 150. Além disso, a quantidade de triplas possíveis, sem gerar quadras, é
    aproximadamente 95. Logo, nossa função de utilidade é a seguinte:
    \vspace{.5em}
    $\textbf{f()} = (d + (150 \times t) + (95 \times q)) + (r \times e)$
    \vspace{.5em}

    Aonde \textbf{$d$} é o número de duplas que a máquina possui, \textbf{$t$} a quantidade de triplas e
    \textbf{$q$} a quantidade de quadras. Além disso, \textbf{$e$} equivale a quantidade de espaços vazios no tabuleiro
    do jogo. Já \textbf{$r$} significa o resultado final do jogo, sendo dado pela seguinte relação:

    \vspace{.5em}
    $
    \textbf{r} = \left\{\begin{array}{lr}
            100, & \text{vitória}\\
            -100, & \text{derrota}\\
            0, & \text{empate}
            \end{array}}
    $

  \subsection {Heurística}
    A cada jogada, uma nova peça é inserida no tabuleiro, sendo ela ou do humano, ou da máquina. Para termos uma
    estimativa que nos informe o quão perto estamos do final do jogo, diversas maneiras foram discutidas e a
    heurística que mais se aproxima da realidade é a seguinte:

    \begin{itemize}
      \item A cada nova jogada, encontra-se as peças pertencentes a máquina
      \item Para cada uma delas, podemos calcular quais sequências pode ser geradas, possuindo espaço entre as peças
      ou não
      \item Soma-se assim a quantidade de duplas, a quantidade de triplas e a quantidade de quadras que a maquina
      possui, atribuindo um peso a cada uma dessas categorias
      \item Ao término desse cálculo, realizamos o mesmo cálculo e descontamos a quantidade de duplas, triplas e
      quadruplas de nosso adversário, utilizando um certo peso
    \end{itemize}

    Ao final desse cálculo, obtemos um valor numérico, utilizado para a tomada de decisão sobre a melhor jogada a ser
    feita em sua próxima rodada. Descontando o número de duplas, triplas e quadras que nosso adversário possui nos
    auxilia a saber se uma jogada é boa ou não para o agente.

    Matematicamente, pode ser representada pela função:

    \vspace{.5em}
    $\textbf{f()} = (d + (150 \times t) + (95 \times q)) + w \times ((d_2 + (150 \times t_2) + (95 \times q_2)))$
    \vspace{.5em}

    Aonde cada uma dessas variaveis possuem o mesmo significado da função de utilidade, aonde $d_2$, $t_2$ e $q_2$
    indicam as sequências do adversário, bem como \textbf{$w$} um peso a ser definido.

\section{Performance e otimizações}
  Para que tenhamos um melhor desempenho ao implementar a heuristica e sua devida função de utilidade, podemos
  focar na busca apenas nas regiões aonde possuímos peças.

  Além disso, podemos sempre tentar realizar a busca através do algoritmo MiniMax com poda sempre pelos nodos com
  maior valor, dado que um maior número de podas será realizado. Assim, reduzimos o espaço de busca e focamos
  somente nos casos que podem nos retornar um resultado melhor.

\end{document}
